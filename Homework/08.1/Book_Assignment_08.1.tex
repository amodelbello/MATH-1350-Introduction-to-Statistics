\documentclass[12pt,fleqn]{article}
\usepackage{
  amsmath,
  amssymb,
  booktabs,
  geometry,
  graphicx,
  microtype,
  parskip,
  caption,
}
\usepackage[shortlabels]{enumitem}

\geometry{margin=3cm}

% equation line spacing
\setlength{\jot}{0.5cm}

% meta data
\newcommand{\chapter}{8.1}
\newcommand{\authorname}{Amo DelBello}
\newcommand{\classdescription}{MATH 1350-D2}
\newcommand{\classname}{Introduction to Statistics, Fall 2022}
\newcommand{\assignment}{\chapter\ Book Assignment}

\newcommand{\problem}[1]{\vspace{5ex}\section*{\chapter-#1}}
\newcommand{\thead}[1]{\textnormal{\textbf{#1}}}
\newcommand{\tvspace}{\vspace{.25cm}}

\title{\classdescription\ \\ \classname\ \\ $\ $ \\ \assignment}
\author{\authorname}
\date{\today}


\begin{document}

\maketitle


\problem{1}
Rejection of the Bayer aspirin claim would have more serious implications because aspirin is a drug. I imagine giving someone the wrong dose of aspirin could cause them adverse health effects. It would probably be safer to use a smaller significance level for hypothesis tests about the mean of aspirin compared to vitamin C.


\problem{2}
An estimate is the process of estimating a population parameter based on what's known about a sample. A hypothesis test is the process of rejecting or not rejecting a hypothesis about a population.


\problem{3}
\begin{enumerate}[label=\alph*.]
\item The null hypothesis is that the mean is equal to 174.1 cm, denoted $H_0:\mu = 174.1\text{cm}$.
\item The alternative hypothesis is that the mean does not equal 174.1 cm, denoted $H_0:\mu \ne 174.1\text{cm}$.
\item The possible conclusions are to reject, or fail to reject the null hypothesis.
\item No. The best you can do is \textit{there is not significant evidence to reject the claim}.
\end{enumerate}


\problem{5}
\begin{enumerate}[label=\alph*.]
\item $p > 0.5$
\item Null hypothesis $H_0: p = 0.5$, Alternative hypothesis $H_1: p > 0.5$
\end{enumerate}


\problem{7}
\begin{enumerate}[label=\alph*.]
\item $\mu = 69 \text{~bpm}$
\item Null hypothesis $H_0: \mu = 69 \text{~bpm}$, Alternative hypothesis $H_1: \mu \ne 69 \text{~bpm}$
\end{enumerate}


\problem{9}
The rate of 59\% is significantly high. The original claim cannot be rejected.

\problem{11}
The rate of 69 bpm does not differ significantly from 69.6 bpm. The original claim cannot be rejected.


\problem{15}
\begin{align*}
  t &= \frac{\bar{x} - \mu}{\frac{s}{\sqrt{n}}} \\
  t &= \frac{69.6 - 69}{\frac{11.3}{\sqrt{153}}} \\
  t &= 0.657
\end{align*}


\problem{20}
\begin{enumerate}[label=\alph*.]
\item Two-tailed test
\item p-value = 0.05238
\item Reject $H_0$
\end{enumerate}


\problem{22}
\begin{enumerate}[label=\alph*.]
\item Critical value: -1.645
\item We should reject $H_0$.
\end{enumerate}


\problem{23}
\begin{enumerate}[label=\alph*.]
\item Critical value: $\pm 1.96$
\item We should reject $H_0$.
\end{enumerate}


\problem{29}
\textbf{Type I Error:} In reality $p = 0.1$, but sample evidence leads us to reject that claim.

\textbf{Type II Error:} In reality $p \ne 0.1$, but we fail to reject the claim that $p = 0.1$


\end{document}
%%% Local Variables:
%%% mode: latex
%%% TeX-master: t
%%% End:
