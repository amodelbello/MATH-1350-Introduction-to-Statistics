\documentclass[12pt,fleqn]{article}
\usepackage{
  amsmath,
  booktabs,
  enumitem,
  geometry,
  graphicx,
  microtype,
  parskip,
}
\geometry{margin=3cm}

% equation line spacing
\setlength{\jot}{0.5cm}

% meta data
\newcommand{\chapter}{3.2}
\newcommand{\authorname}{Amo DelBello}
\newcommand{\classdescription}{MATH 1350-D2}
\newcommand{\classname}{Introduction to Statistics, Fall 2022}
\newcommand{\assignment}{\chapter\ Book Assignment}

\newcommand{\problem}[1]{\vspace{5ex}\section*{\chapter-#1}}
\newcommand{\thead}[1]{\textnormal{\textbf{#1}}}
\newcommand{\tvspace}{\vspace{.25cm}}

\title{\classdescription\ \\ \classname\ \\ $\ $ \\ \assignment}
\author{\authorname}
\date{\today}


\begin{document}
\maketitle

\problem{5}
\begin{description}
\item[Range:] $\\$
  \textbf{Range = (max data value) \-- (min data value)} \\
  \[
    99 - 7 = 92.0
  \]

\item[Standard Deviation:] $\\$
  \begin{align*}
    s &= \sqrt{\frac{n(\Sigma{x^2}) - (\Sigma{x})^2}{n(n-1)}} \\
    s &= \sqrt{\frac{11(47348) - 628^2}{11(11-1)}} \\
    s &= \sqrt{\frac{126444}{110}} \\
    s &= 33.9
  \end{align*}

\item[Variance:] $\\$
  \textbf{Variance = $s^2$}
  \[
    33.904142948774108^2 = 1149.5
  \]
\end{description}

As the jersey numbers are purely symbolic nominal data, the range, standard deviation, and variance do not have any meaning or significance.


\problem{8}
\begin{description}
\item[Range:] $\\$
  \textbf{Range = (max data value) \-- (min data value)} \\
  \[
    264 - 77 = 187.0~\text{USD}
  \]

\item[Standard Deviation:] $\\$
  \begin{align*}
    s &= \sqrt{\frac{n(\Sigma{x^2}) - (\Sigma{x})^2}{n(n-1)}} \\
    s &= \sqrt{\frac{8(266996) - 1370^2}{8(8-1)}} \\
    s &= \sqrt{\frac{259068}{56}} \\
    s &= 68.0~\text{USD}
  \end{align*}

\item[Variance:] $\\$
  \textbf{Variance = $s^2$}
  \[
    68.01627956389768^2 = 4626.2~\text{USD}^2
  \]
\end{description}
The measures of variation for the Las Vegas room prices sample may be useful for someone if they are trying to find the best price. A high standard deviation could indicate that you may be paying way too much or that a really good deal exists and you just have to find it.


\problem{12}
\begin{description}
\item[Range:] $\\$
  \textbf{Range = (max data value) \-- (min data value)} \\
  \[
    1.49 - 0.51 = 0.980~W/kg
  \]

\item[Standard Deviation:] $\\$
  \begin{align*}
    s &= \sqrt{\frac{n(\Sigma{x^2}) - (\Sigma{x})^2}{n(n-1)}} \\
    s &= \sqrt{\frac{11(16.407799999999998) - 12.959999999999999^2}{11(11-1)}} \\
    s &= \sqrt{\frac{12.524200000000008}{110}} \\
    s &= 0.337~W/kg
  \end{align*}

\item[Variance:] $\\$
  \textbf{Variance = $s^2$}
  \[
    0.33742608618238706^2 = 0.113~(W/kg)^2
  \]
\end{description}
If one of each model of cell phone is measured for radiation and the results are used to find the measures of variation, the results probably are not typical of the population of cell phones that are in use because the popularity of different brands varies widely.


\problem{13}
\begin{description}
\item[Range:] $\\$
  \textbf{Range = (max data value) \-- (min data value)} \\
  \[
    55 - 0 = 55.0~mg
  \]

\item[Standard Deviation:] $\\$
  \begin{align*}
    s &= \sqrt{\frac{n(\Sigma{x^2}) - (\Sigma{x})^2}{n(n-1)}} \\
    s &= \sqrt{\frac{20(29045) - 651^2}{20(20-1)}} \\
    s &= \sqrt{\frac{157099}{380}} \\
    s &= 20.3~mg
  \end{align*}

\item[Variance:] $\\$
  \textbf{Variance = $s^2$}
  \[
    20.33269340379261^2 = 413.4~mg^2
  \]
\end{description}
As some brands are consumed more than others, perhaps by a large amount, it's not safe to assume that the statistics are representative of the population of \textit{all cans} for the 20 brands.


\problem{21}
\begin{description}
\item[Standard Deviation of Systolic Blood Pressure:]
  \begin{align*}
    s &= \sqrt{\frac{n(\Sigma{x^2}) - (\Sigma{x})^2}{n(n-1)}} \\
    s &= \sqrt{\frac{10(165936) - 1276^2}{10(10-1)}} \\
    s &= \sqrt{\frac{21184}{90}} \\
    s &= 18.6~mm~Hg
  \end{align*}

\item[Coefficient of Variation of Systolic Blood Pressure:]
  \begin{align*}
    CV &= \frac{s}{\bar{x}} * 100 \\
    CV &= \frac{18.6~mm~Hg}{127.6} * 100 \\
    CV &= 14.6~mm~Hg
  \end{align*}

\item[Standard Deviation of Diastolic Blood Pressure:]
  \begin{align*}
    s &= \sqrt{\frac{n(\Sigma{x^2}) - (\Sigma{x})^2}{n(n-1)}} \\
    s &= \sqrt{\frac{10(55568) - 736^2}{10(10-1)}} \\
    s &= \sqrt{\frac{13984}{90}} \\
    s &= 12.5~mm~Hg
  \end{align*}

\item[Coefficient of Variation of Diastolic Blood Pressure:]
  \begin{align*}
    CV &= \frac{s}{\bar{x}} * 100 \\
    CV &= \frac{12.5~mm~Hg}{73.6} * 100 \\
    CV &= 16.9~mm~Hg
  \end{align*}

\item[Comparison of Coefficient of Variations:] $\\$
The coefficient of variation of diastolic blood pressure is just slightly higher than that of systolic blood pressure. This is interesting to me because the opposite is true of their respective standard deviations.
\end{description}


\problem{27}
\begin{description}
\item[Range:] $\\$
  \textbf{Range = (max data value) \-- (min data value)} \\
  \[
    99.6 - 96.5 = 3.10^\circ F
  \]

\item[Standard Deviation:] $\\$
  \begin{align*}
    s &= \sqrt{\frac{n(\Sigma{x^2}) - (\Sigma{x})^2}{n(n-1)}} \\
    s &= \sqrt{\frac{106(1022224.1799999992) - 10409.2^2}{106(106-1)}} \\
    s &= \sqrt{\frac{4318.439999908209}{11130}} \\
    s &= 0.62^\circ F
  \end{align*}

\item[Variance:] $\\$
  \textbf{Variance = $s^2$}
  \[
    0.6228964600892775^2 = 0.39~(^\circ F)^2
  \]
\end{description}


\problem{39}
\begin{description}
\item[Midpoints \& Frequency:]
  \begin{align*}
    data = [ \\
      &\{\text{'midpoint': 49.5, 'frequency': 29\}}, \\
      &\{\text{'midpoint': 149.5, 'frequency': 34\}}, \\
      &\{\text{'midpoint': 249.5, 'frequency': 14\}}, \\
      &\{\text{'midpoint': 349.5, 'frequency': 3\}}, \\
      &\{\text{'midpoint': 449.5, 'frequency': 5\}}, \\
      &\{\text{'midpoint': 549.5, 'frequency': 1\}}, \\
      &\{\text{'midpoint': 649.5, 'frequency': 1\}}, \\
    ]
  \end{align*}

\item[Standard Deviation for a Frequency Distribution:]
  \begin{align*}
    s &= \sqrt{\frac{n[\Sigma(f * x^2)] - [\Sigma(f * x)]^2}{n(n-1)}} \\
    s &= \sqrt{\frac{153[8388188.25] - 34273.5^2}{23256}} \\
    s &= 68.4
  \end{align*}

\end{description}
The actual standard deviation of 59.5 differs considerably from standard deviation of the frequency distribution calculated to be 68.4


\problem{40}
\begin{description}
\item[Midpoints \& Frequency:]
  \begin{align*}
    data = [ \\
      &\{\text{'midpoint': 149.5, 'frequency': 25\}}, \\
      &\{\text{'midpoint': 249.5, 'frequency': 92\}}, \\
      &\{\text{'midpoint': 349.5, 'frequency': 28\}}, \\
      &\{\text{'midpoint': 449.5, 'frequency': 0\}}, \\
      &\{\text{'midpoint': 549.5, 'frequency': 2\}}, \\
    ]
  \end{align*}

\item[Standard Deviation for a Frequency Distribution:]
  \begin{align*}
    s &= \sqrt{\frac{n[\Sigma(f * x^2)] - [\Sigma(f * x)]^2}{n(n-1)}} \\
    s &= \sqrt{\frac{147[10309886.75] - 37576.5^2}{21462}} \\
    s &= 69.5
  \end{align*}

\end{description}
The actual standard deviation of 65.4 is fairly close to the standard deviation of the frequency distribution calculated to be 69.5

\end{document}
%%% Local Variables:
%%% mode: latex
%%% TeX-master: t
%%% End:
