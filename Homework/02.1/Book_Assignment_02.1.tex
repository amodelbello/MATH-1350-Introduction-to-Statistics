\documentclass[fleqn, 12pt]{article}
\usepackage[letterpaper, inner=3cm, outer=3cm, top=3cm, bottom=3cm]{geometry}
\usepackage{
  amsmath,
  booktabs,
  enumitem,
  microtype,
  parskip,
}

% meta data
\newcommand{\chapter}{2.1}
\newcommand{\authorname}{Amo DelBello}
\newcommand{\classdescription}{MATH 1350-D2}
\newcommand{\classname}{Introduction to Statistics, Fall 2022}
\newcommand{\assignment}{\chapter\ Book Assignment}

\newcommand{\problem}[1]{\vspace{5ex}\section*{\chapter-#1}}
\newcommand{\thead}[1]{\textnormal{\textbf{#1}}}
\newcommand{\tvspace}{\vspace{.25cm}}

\title{\classdescription\ \\ \classname\ \\ $\ $ \\ \assignment}
\author{\authorname}
\date{\today}


\begin{document}
\maketitle

\problem{6}
\begin{description}
\item[class width:] $30 - 20 = 10$
\item[class midpoints:] \phantom{1}
  \[
    \begin{split}
    (20 + 29) / 2 = 24.5 \\
    (30 + 39) / 2 = 34.5 \\
    (40 + 49) / 2 = 44.5 \\
    (50 + 59) / 2 = 54.5 \\
    (60 + 69) / 2 = 64.5 \\
    (70 + 79) / 2 = 74.5 \\
  \end{split}
  \]
\item[class boundaries:] 19.5, 29.5, 39.5, 49.5, 59.5, 69.5, 79.5
\item[number of individuals:] $1 + 28 + 36 + 15 + 6 + 1 = 87$

\end{description}


\problem{7}
\begin{description}
\item[class width:] $100 - 0 = 100$
\item[class midpoints:] \phantom{1}
  \[
  \begin{split}
  (0 + 99) / 2 &= 49.5 \\
  (100 + 199) / 2 &= 149.5 \\
  (200 + 299) / 2 &= 249.5 \\
  (300 + 399) / 2 &= 349.5 \\
  (400 + 499) / 2 &= 449.5 \\
  (500 + 599) / 2 &= 549.5 \\
  (600 + 699) / 2 &= 649.5
  \end{split}
  \]
\item[class boundaries:] -0.5, 99.5, 199.5, 299.5, 399.5, 499.5, 599.5, 699.5
\item[number of individuals:] $1 + 51 + 90 + 10 + 0 + 0 + 1 = 153$
\end{description}

\problem{12}
\begin{tabular}{@{}ll@{}}
  \thead{Intensity} & \thead{Frequency} \\
  \toprule
  0 & 24 \\
  1 & 16 \\
  2 & 2 \\
  3 & 2 \\
  4 & 1 \\
  \bottomrule
\end{tabular}


\problem{18}
\begin{tabular}{@{}ll@{}}
  \thead{Last Digit} & \thead{Frequency} \\
  \toprule
  0 & 26 \\
  1 & 1 \\
  2 & 1 \\
  3 & 2 \\
  4 & 2 \\
  5 & 12 \\
  6 & 1 \\
  7 & 0 \\
  8 & 4 \\
  9 & 1 \\
  \bottomrule
\end{tabular}
\tvspace\
\begin{itemize}
  \item Because the results are overwhelmingly round numbers, i.e. 0 and 5, it can be assumed that most of the data comes from estimates and not accurate measurements.
  \item We can assume the results of the survey are less accurate than if exact, measured weights were given.
\end{itemize}


\problem{20}
\begin{tabular}{@{}*{3}r@{}}
  \thead{Platelet Count} & \thead{Men} & \thead{Women} \\
  \toprule
  0\--99    & 1\% &  \\
  100\--199 & 33\% & 17\% \\
  200\--299 & 58\% & 63\% \\
  300\--399 & 7\% & 19\% \\
  400\--499 & 0\% & 0\% \\
  500\--599 & 0\% & 1\% \\
  600\--699 & 1\% &  \\
  \bottomrule
\end{tabular}
\tvspace\

The counts for both men and women are similar. They both conform to a normal distribution. The women appear to have slightly higher platelet counts but the difference may be attributed to chance.


\problem{25}
\begin{tabular}{@{}p{5cm}l@{}}
  \thead{Systolic Blood Pressure (mm Hg)} & \thead{Frequency} \\
  \toprule
  80\--99 & 11 \\
  100\--119 & 116 \\
  120\--139 & 131 \\
  140\--159 & 34 \\
  160\--179 & 7 \\
  180\--199 & 1 \\
  \bottomrule
\end{tabular}
\tvspace\

The frequency distribution appears to be a normal distribution skewed to the right.

\end{document}
%%% Local Variables:
%%% mode: latex
%%% TeX-master: t
%%% End:
