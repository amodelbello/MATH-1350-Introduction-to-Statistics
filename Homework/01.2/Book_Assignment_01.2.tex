\documentclass{article}
\usepackage{enumitem}

% Don't indent paragraphs
\setlength\parindent{0pt}

% meta data
\newcommand{\chapter}{1.2}
\newcommand{\authorname}{Amo DelBello}
\newcommand{\classdescription}{MATH 1350-D2}
\newcommand{\classname}{Introduction to Statistics, Fall 2022}
\newcommand{\assignment}{\chapter\ Book Assignment}

% headers
\newcommand{\problem}[1]{\vspace{5ex}\section*{\chapter-#1}}
\newcommand{\solution}{\subsection*{Solution}}

\title{\classdescription\ \\ \classname\ \\ $\ $ \\ \assignment}
\author{\authorname}
\date{\today}

\begin{document}
\maketitle


\problem{1}
The population is adults in the United States. The sample is the 2276 adults surveyed. The value of $33\%$ is a statistic.


\problem{2}
\begin{enumerate}[label=\textbf{\alph*.}]
\item Quantitative Data
\item Categorical Data (There aren't any cigarette brands in this data set. If there were, they would be categorical.)
\item Categorical Data
\item Quantitative Data
\end{enumerate}


\problem{3}
Only \textbf{(a)} is an example of discrete data.


\problem{4}
\begin{enumerate}[label=\textbf{\alph*.}]
\item The sample is the 1020 adults included in the survey. The population is all adults in the United States.
\item The value of $44\%$ is a statistic.
\item The level of measurement of the value $44\%$ is \textit{ratio} because there is a natural zero and ratios make sense
\item The numbers of subjects in such surveys are discrete.
\end{enumerate}


\problem{5}
The given value is a statistic.


\problem{10}
The given value is a statistic


\problem{15}
The data is from a discrete data set


\problem{20}
The data is from a discrete data set


\problem{25}
The \textit{interval} level is most appropriate because differences in years are meaningful but there is no natural zero starting point and ratios are meaningless.


\problem{30}
The level of measurement is \textit{ordinal} because the data can be arranged in some order but the differences between the data values are meaningless. In addition to the calculation being meaningless, there are privacy issues with requiring students to report parts of their social security numbers.


\end{document}

%%% Local Variables:
%%% mode: latex
%%% TeX-master: t
%%% End:
