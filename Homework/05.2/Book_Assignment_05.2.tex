\documentclass[12pt,fleqn]{article}
\usepackage{
  amsmath,
  amssymb,
  booktabs,
  geometry,
  graphicx,
  microtype,
  parskip,
}
\usepackage[shortlabels]{enumitem}

\geometry{margin=3cm}

% equation line spacing
\setlength{\jot}{0.5cm}

% meta data
\newcommand{\chapter}{5.2}
\newcommand{\authorname}{Amo DelBello}
\newcommand{\classdescription}{MATH 1350-D2}
\newcommand{\classname}{Introduction to Statistics, Fall 2022}
\newcommand{\assignment}{\chapter\ Book Assignment}

\newcommand{\problem}[1]{\vspace{5ex}\section*{\chapter-#1}}
\newcommand{\thead}[1]{\textnormal{\textbf{#1}}}
\newcommand{\tvspace}{\vspace{.25cm}}

\title{\classdescription\ \\ \classname\ \\ $\ $ \\ \assignment}
\author{\authorname}
\date{\today}


\begin{document}

\maketitle

\problem{6}
The procedure results in a binomial distribution


\problem{8}
The procedure does not result in a binomial distribution because the outcomes are not classified into exactly two categories.


\problem{11}
The procedure results in a distribution that can be treated as binomial, because the sample size is small enough in relation to the population that it can be used without replacement.


\problem{17}
\begin{align*}
  P(x) &= \frac{n!}{(n-x)!x!} * p^x * q^{n-x} \\
\\
  P(2) &= \frac{8!}{(8-2)!2!} * 0.2^2 * 0.8^{8-2} \\
  P(2) &= 0.29360128 \\
\\
  P(1) &= \frac{8!}{(8-1)!1!} * 0.2^1 * 0.8^{8-1} \\
  P(1) &= 0.33554432 \\
\\
  P(0) &= \frac{8!}{(8-0)!0!} * 0.2^0 * 0.8^{8-0} \\
  P(0) &= 0.16777216 \\
\\
  P(0|1|2) &= 0.16777216 + 0.33554432 + 0.29360128 \\
  P(0|1|2) &= 0.797
\end{align*}


\problem{18}
$(< 3)~\equiv~(\le 2)~\therefore~P(\le{2}) = 0.797$


\problem{29}
\begin{enumerate}[label=\alph*.]
\item
  \begin{align*}
    \mu = np = (36)(0.5) = 18.0~\text{girls} \\
    \sigma = \sqrt{npq} = \sqrt{(36)(0.5)(0.5)} = 3.0~\text{girls}
  \end{align*}
\item
  \begin{align*}
    \mu - 2\sigma = 18 - 2(3) = 12.0~\text{girls} \\
    \mu + 2\sigma = 18 + 2(3) = 24.0~\text{girls} \\
  \end{align*}
\item
  The result of 26 girls is significantly high because it is greater than two standard deviations. It suggests the XSORT method is effective.
\end{enumerate}


\problem{30}
\begin{enumerate}[label=\alph*.]
\item
  \begin{align*}
    \mu = np = (16)(0.5) = 8.0~\text{girls} \\
    \sigma = \sqrt{npq} = \sqrt{(16)(0.5)(0.5)} = 2.0~\text{girls}
  \end{align*}
\item
  \begin{align*}
    \mu - 2\sigma = 8 - 2(2) = 4.0~\text{girls} \\
    \mu + 2\sigma = 8 + 2(2) = 12.0~\text{girls} \\
  \end{align*}
\item
  The result of 11 girls is not significantly high because it is within two standard deviations. It suggests the XSORT method is not effective.
\end{enumerate}


\end{document}
%%% Local Variables:
%%% mode: latex
%%% TeX-master: t
%%% End:
