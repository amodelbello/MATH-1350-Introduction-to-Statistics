\documentclass[12pt,fleqn]{article}
\usepackage{
  amsmath,
  booktabs,
  geometry,
  graphicx,
  microtype,
  parskip,
}
\usepackage[shortlabels]{enumitem}

\geometry{margin=3cm}

% equation line spacing
\setlength{\jot}{0.5cm}

% meta data
\newcommand{\chapter}{4.1}
\newcommand{\authorname}{Amo DelBello}
\newcommand{\classdescription}{MATH 1350-D2}
\newcommand{\classname}{Introduction to Statistics, Fall 2022}
\newcommand{\assignment}{\chapter\ Book Assignment}

\newcommand{\problem}[1]{\vspace{5ex}\section*{\chapter-#1}}
\newcommand{\thead}[1]{\textnormal{\textbf{#1}}}
\newcommand{\tvspace}{\vspace{.25cm}}

\title{\classdescription\ \\ \classname\ \\ $\ $ \\ \assignment}
\author{\authorname}
\date{\today}


\begin{document}
\maketitle

\problem{1}
\begin{align*}
  P(A) &= \frac{1}{1000} \\
  P(\bar{A}) &= \frac{999}{1000} \\
\end{align*}


\problem{2}
The probability of rain today is $\frac{1}{4}$, or $P(A) = 0.25$.


\problem{3}
The correct answer is \textbf{c}: 0.60 probably that it will rain somewhere in the author's region at some point during the day.


\problem{4}
I'd say maybe I successfully turn on a single light 3 time a day in my house. Light bulbs tend to last for several years, maybe like 5 years. So switching on a light 3 times a day for 5 years would be 5475 or around 5500.

So the probability that the next time I turn on a light switch, I discover that a bulb does work would be:
\begin{equation*}
  P(\bar{A}) = \frac{5499}{5500}
\end{equation*}


\problem{5}
Among the items listed, $0, \frac{3}{5}, 1, \text{and}~ 0.135$ are probabilities.


\problem{9}
(b) significantly high


\problem{10}
(c) neither significantly low nor significantly high


\problem{11}
(c) neither significantly low nor significantly high


\problem{12}
(a) significantly low


\problem{21}
There are 5 total false negatives in the entire sample of 555 subjects. If 1 of the total subjects was selected then the probability of selecting a subject that was a false negative would be:
\begin{align*}
  P(A) &= \frac{5}{555} \\
  P(A) &= 0.00901
\end{align*}
The employer would suffer because they could possibly hire a drug user when their intent was to hire non-drug users.


\problem{22}
There are 25 total false positives in the entire sample of 555 subjects. If 1 of the total subjects was selected then the probability of selecting a subject that was a false positive would be:
\begin{align*}
  P(A) &= \frac{25}{555} \\
  P(A) &= 0.045
\end{align*}
The job applicant would suffer from a false positive because they would be denied employment for being a drug user when they are not a drug user.


\problem{23}
There are 50 (45 true positive and 5 false negative) drug users in the sample. The probability of selecting a drug user would be:
\begin{align*}
  P(A) &= \frac{50}{555} \\
  P(A) &= 0.0901
\end{align*}
The result does appear to be reasonable as an estimate of the ``prevalence rate'' described in the Chapter Problem.


\problem{24}
There are 505 (480 true negative and 25 false positive) non-drug users in the sample. The probability of selecting a non-drug user would be:
\begin{align*}
  P(A) &= \frac{505}{555} \\
  P(A) &= 0.91
\end{align*}
The result does appear to be reasonable as an estimate of the ``prevalence rate'' described in the Chapter Problem, but rather the \textit{compliment} of the prevalence rate.


\problem{30}
The probability that a randomly selected passenger car crash results in a rollover can be expressed as:
\begin{align*}
  P(A) &= \frac{83600}{83600 + 5127400} \\
  P(A) &= 0.016
\end{align*}
Since less than 1 in 50 crashes resulted in a rollover, it is unlikely for a car to roll over in a crash.


\problem{37}
I conclude that chance cannot be rejected as a reasonable explanation and, based on the study, we cannot reasonably assert pregnant women have the ability to predict the sex of their baby.


\problem{38}
I conclude that at a 0.0001 probability, chance should be rejected as a reasonable explanation for the difference in survival rates between seatbelt wearers and non-seatbelt wearers in head-on collisions, and that wearing a seatbelt improves the survival rate in head-on collisions.


\end{document}
%%% Local Variables:
%%% mode: latex
%%% TeX-master: t
%%% End:
