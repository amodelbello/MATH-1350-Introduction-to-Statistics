\documentclass[12pt,fleqn]{article}
\usepackage{
  amsmath,
  booktabs,
  enumitem,
  geometry,
  graphicx,
  microtype,
  parskip,
}
\geometry{margin=3cm}

% meta data
\newcommand{\chapter}{3.1}
\newcommand{\authorname}{Amo DelBello}
\newcommand{\classdescription}{MATH 1350-D2}
\newcommand{\classname}{Introduction to Statistics, Fall 2022}
\newcommand{\assignment}{\chapter\ Book Assignment}

\newcommand{\problem}[1]{\vspace{5ex}\section*{\chapter-#1}}
\newcommand{\thead}[1]{\textnormal{\textbf{#1}}}
\newcommand{\tvspace}{\vspace{.25cm}}

\title{\classdescription\ \\ \classname\ \\ $\ $ \\ \assignment}
\author{\authorname}
\date{\today}


\begin{document}
\maketitle

\enlargethispage{\baselineskip}

\problem{5}
\begin{tabular}{@{}ll@{}}
  \thead{Jersey Numbers} \\
  \toprule
  Mean & 57.1 \\
  Median & 60 \\
  Midrange & 53.0 \\
  Mode & There is no mode \\
  \bottomrule
\end{tabular}
\vspace{1em}

The results tell us nothing about the arbitrary numeric labels assigned to players.


\problem{6}
\begin{tabular}{@{}ll@{}}
  \thead{Player Weights} \\
  \toprule
  Mean & 231.1 \\
  Median & 225 \\
  Midrange & 247.0 \\
  Mode & 190 \\
  \bottomrule
\end{tabular}
\vspace{1em}

No. The average weights from a sample of a single team cannot be considered to be representative of the entire league. This would be essentially taking a stratified sample with only a single stratum.


\problem{10}
\begin{tabular}{@{}ll@{}}
  \thead{Phenotype Codes} \\
  \toprule
  Mean & 1.9 \\
  Median & 2 \\
  Midrange & 2.5 \\
  Mode & 1 \\
  \bottomrule
\end{tabular}
\vspace{1em}

The only measure of center that makes sense here, i.e.\ with nominal data, is the \textit{mode} of the data which is $1$ because it occurs 11 times, more than any other value.


\problem{13}
\begin{tabular}{@{}ll@{}}
  \thead{Caffeine} \\
  \toprule
  Mean & 32.6 \\
  Median & 39.5 \\
  Midrange & 27.5 \\
  Mode & 0 \\
  \bottomrule
\end{tabular}
\vspace{1em}

As some brands are consumed more than others, perhaps by a large amount, it's not safe to assume that the statistics are representative of the population of \textit{all cans} for the 20 brands.


\problem{18}
\begin{tabular}{@{}ll@{}}
  \thead{Record Sales} \\
  \toprule
  Mean & 1.80 \\
  Median & 1.3 \\
  Midrange & 3.20 \\
  Mode & 1.4 \\
  \bottomrule
\end{tabular}
\vspace{1em}

The measures of center do not give us any insight into a changing trend over time. They only tell us the ``middle'' values of the sample as a whole.


\problem{23}
\begin{tabular}{@{}lll@{}}
  & \thead{Male} & \thead{Female} \\
  \toprule
  Mean & 69.5 & 82.1 \\
  Median & 66 & 84 \\
  \bottomrule
\end{tabular}
\vspace{1em}

The heart rates males appear to be lower than those of females by about 12 beats per minute.


\problem{25}
\begin{tabular}{@{}ll@{}}
  \thead{Tornado F-Scales} \\
  \toprule
  Mean & 0.8 \\
  Median & 1 \\
  \bottomrule
\end{tabular}
\vspace{1em}

There are $10$ missing F-Scale measurements in the data set.


\problem{28}
I \textit{guess} from the vague wording of the question that I am expected to ``make an observation'' about the fact that the birth weight values are all whole numbers. Therefore the results should be rounded to a single decimal place.


\pagebreak
\problem{32}
% * Find the mean summarized in the frequency distribution
% * Compare to the actual mean

Mean from a frequency distribution:
\[
  \bar{x} = \frac{\Sigma (f * x)}{\Sigma f}
\]

The frequencies are: $[25, 92, 28, 0, 2]$

The midpoints are: $[149.5, 249.5, 349.5, 449.5, 549.5]$

The mean of the data summarized in the frequency distribution is:
\begin{align}
  \bar{x} &= \frac{(25* 149.5) + (92 * 249.5) + (28 * 349.5) + (0 * 449.5) + (2 * 549.5)}{(25 + 92 + 28 + 0 + 2)} \\
  \bar{x} &= \frac{3737.5 + 22954 + 9786 + 0 + 1099}{147} \\
  \bar{x} &= 255.62
\end{align}

The mean obtained from using the summary in the frequency distribution \textbf{255.62} is similar but slightly greater than the actual mean from the original data list which was \textbf{255.1}. Also the result from the frequency distribution is rounded to two decimal places because the data used to compute the mean all had single decimal places. I'm surprised that the mean given by the question has only a single decimal place since the original platelet data has a single decimal place.

\end{document}
%%% Local Variables:
%%% mode: latex
%%% TeX-master: t
%%% End:
