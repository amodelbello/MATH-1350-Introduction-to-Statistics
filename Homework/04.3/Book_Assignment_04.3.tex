\documentclass[12pt,fleqn]{article}
\usepackage{
  amsmath,
  booktabs,
  geometry,
  graphicx,
  microtype,
  parskip,
}
\usepackage[shortlabels]{enumitem}

\geometry{margin=3cm}

% equation line spacing
\setlength{\jot}{0.5cm}

% meta data
\newcommand{\chapter}{4.3}
\newcommand{\authorname}{Amo DelBello}
\newcommand{\classdescription}{MATH 1350-D2}
\newcommand{\classname}{Introduction to Statistics, Fall 2022}
\newcommand{\assignment}{\chapter\ Book Assignment}

\newcommand{\problem}[1]{\vspace{5ex}\section*{\chapter-#1}}
\newcommand{\thead}[1]{\textnormal{\textbf{#1}}}
\newcommand{\tvspace}{\vspace{.25cm}}

\title{\classdescription\ \\ \classname\ \\ $\ $ \\ \assignment}
\author{\authorname}
\date{\today}


\begin{document}

\maketitle

\problem{1}
$\bar{A}$ is the event of not getting any defective iPhones when 3 are randomly selected with replacement from a batch.


\problem{2}
Both \textbf{(a)} and \textbf{(b)} are correct.


\problem{6}
The couple is equally likely to have a boy or a girl, regardless of how many children they already have.


\problem{9}
\begin{align*}
  P(\bar{A}) &= 0.9^4 \\
  P(\bar{A}) &= 0.6561 \\
  P(A) &= 1 - 0.6561 \\
  P(A) &= 0.344
\end{align*}


\problem{12}
\begin{align*}
  P(\bar{A}) &= {(1 - 0.67)}^4 \\
  P(\bar{A}) &= 0.0119 \\
  P(A) &= 1 - 0.0119 \\
  P(A) &= 1 - 0.0119 \\
  P(A) &= 0.989
\end{align*}
The result is not affected at all by the additional information that the survey subjects volunteered to respond.


\problem{13}
\begin{enumerate}[label=\alph*.]
\item
  \begin{align*}
    P(A) &= \frac{27}{27 + 16} \\
    P(A) &= 0.628
  \end{align*}

\item
  \begin{align*}
    P(A) &= \frac{16}{27 + 16} \\
    P(A) &= 0.372
  \end{align*}

\item The preceding results suggest that when given quarters, students are more likely to purchase gum than keep it.
\end{enumerate}


\problem{14}
\begin{enumerate}[label=\alph*.]
\item
  \begin{align*}
    P(A) &= \frac{12}{12 + 34} \\
    P(A) &= 0.261
  \end{align*}

\item
  \begin{align*}
    P(A) &= \frac{34}{12 + 34} \\
    P(A) &= 0.739
  \end{align*}

\item The preceding results suggest that when given a \$1 bill, students are more likely to keep it than purchase gum.
\end{enumerate}


\problem{15}
\begin{enumerate}[label=\alph*.]
\item
  ??? This is the same as 13(a)
  \begin{align*}
    P(A) &= \frac{27}{27 + 16} \\
    P(A) &= 0.628
  \end{align*}

\item
   ??? This is the same as 14(a)
  \begin{align*}
    P(A) &= \frac{12}{12 + 34} \\
    P(A) &= 0.261
  \end{align*}

\item The preceding results suggest that a student is more likely to spend money on gum when given quarters, and less likely to spend it when given a \$1 dollar bill.
\end{enumerate}


\problem{16}
\begin{enumerate}[label=\alph*.]
\item
  (okay I guess I'll just go with it then)
  \begin{align*}
    P(A) &= \frac{16}{27 + 16} \\
    P(A) &= 0.372
  \end{align*}

\item
  \begin{align*}
    P(A) &= \frac{34}{12 + 34} \\
    P(A) &= 0.739
  \end{align*}

\item The preceding results suggest that a student is more likely to keep the money when given a \$1 dollar bill, and less likely to keep it when given quarters.
\end{enumerate}


\problem{17}
\begin{align*}
  P(A) &= \frac{2}{2 + 1155} \\
  P(A) &= 0.00173
\end{align*}
This case is problematic for test subjects because they will think they suffer from a disease they don't actually have. Thankfully, it would appear these cases are fairly rare.


\problem{18}
\begin{align*}
  P(A) &= \frac{10}{10 + 335} \\
  P(A) &= 0.0290
\end{align*}
Unfavorable consequences of this error are test subjects who consider themselves healthy, actually suffer from Hepatitis C.


\problem{19}
\begin{align*}
  P(A) &= \frac{335}{2 + 335} \\
  P(A) &= 0.994
\end{align*}
Based on the high result, the test appears to be effective.

\problem{20}
\begin{align*}
  P(A) &= \frac{1153}{10 + 1153} \\
  P(A) &= 0.991
\end{align*}
Based on the high result, the test also appears to be effective.


\end{document}
%%% Local Variables:
%%% mode: latex
%%% TeX-master: t
%%% End:
