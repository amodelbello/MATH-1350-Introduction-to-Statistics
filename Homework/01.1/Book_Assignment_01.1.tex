\documentclass{article}
\usepackage{enumitem}

% Don't indent paragraphs
\setlength\parindent{0pt}

% meta data
\newcommand{\chapter}{1.1}
\newcommand{\authorname}{Amo DelBello}
\newcommand{\classdescription}{MATH 1350-D2}
\newcommand{\classname}{Introduction to Statistics, Fall 2022}
\newcommand{\assignment}{\chapter\ Book Assignment}

% headers
\newcommand{\problem}[1]{\vspace{5ex}\section*{\chapter-#1}}
\newcommand{\solution}{\subsection*{Solution}}

\title{\classdescription\ \\ \classname\ \\ $\ $ \\ \assignment}
\author{\authorname}
\date{\today}

\begin{document}
\maketitle

\problem{3}
Statistical significance is achieved when we get a result that is very unlikely to occur by chance. Practical significance is achieved when the result or finding make enough of a difference to justify its use. A treatment can be statistically significant while not meeting the threshold of being practically significant.


\problem{6}
The given source (FDA) is not likely to have potential to create bias because it is a government agency that would not benefit from the result of a statistical study.


\problem{10}
The sampling method is flawed because it is self selected.


\problem{15}
As the threshold for statistical significance of an event occurring by chance is around 5\%, at 19\%, these results do not appear to have statistical significance. As they do not have statistical significance they almost certainly do not have practical significance.


\problem{20}
We should conclude that there are no meaningful statistical differences to be found in the body temperature results.


\problem{25}
There is a problem with the source of the data. The poll was sponsored by an organization with a special interest in its results.


\problem{30}
\begin{enumerate}[label=\textbf{\alph*.}]
\item $347 * .73 = 253.31$
\item No because $.31$ of a person does not exist
\item $254$
  \item $112 / 347 = .32276657 = 32.28\%$
\end{enumerate}


\problem{35}
The statement is problematic because $100\%$ of some quantity is \textit{all} of it. If there are references made to percentages that exceed $100\%$, such references are often not justified.


\end{document}
%%% Local Variables:
%%% mode: latex
%%% TeX-master: t
%%% End:
