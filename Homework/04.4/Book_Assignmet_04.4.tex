\documentclass[12pt,fleqn]{article}
\usepackage{
  amsmath,
  booktabs,
  geometry,
  graphicx,
  microtype,
  parskip,
}
\usepackage[shortlabels]{enumitem}

\geometry{margin=3cm}

% equation line spacing
\setlength{\jot}{0.5cm}

% meta data
\newcommand{\chapter}{4.4}
\newcommand{\authorname}{Amo DelBello}
\newcommand{\classdescription}{MATH 1350-D2}
\newcommand{\classname}{Introduction to Statistics, Fall 2022}
\newcommand{\assignment}{\chapter\ Book Assignment}

\newcommand{\problem}[1]{\vspace{5ex}\section*{\chapter-#1}}
\newcommand{\thead}[1]{\textnormal{\textbf{#1}}}
\newcommand{\tvspace}{\vspace{.25cm}}

\title{\classdescription\ \\ \classname\ \\ $\ $ \\ \assignment}
\author{\authorname}
\date{\today}


\begin{document}

\maketitle

\problem{5}
There are 10 different possibilities for each digit, so the total number of different possible pin numbers is $n_1 * n_2 * n_3 * n_4 = 10 * 10 * 10 * 10 = 10,000$. Because all of the pin numbers are equally likely, the probability of getting the correct pin number on the first attempt is:
\begin{align*}
\frac{1}{10,000}
\end{align*}


\problem{6}
Since we are only dealing with the first 5 numbers of the social security number, the total number of different possibilities is $n_1 * n_2 * n_3 * n_4 * n_5 = 10 * 10 * 10 * 10 * 10 = 100,000$. Because all of the pin numbers are equally likely, the probability of getting the correct numbers on the first attempt is
\begin{align*}
\frac{1}{100,000}
\end{align*}


\problem{11}
\begin{align*}
~_nP_r &= \frac{n!}{(n - r)!} \\
~_nP_r &= \frac{50!}{(50 - 5)!} \\
~_nP_r &= \frac{3.041 * 10^{64}}{1.196 * 10^{56}} \\
~_nP_r &= 254,251,200
\end{align*}
The chances of the 5 cities being selected in the correct order are:
\begin{align*}
  \frac{1}{254,251,200}
\end{align*}


\problem{19}
Since we are dealing with 5 numbers of a zip code, the total number of different possibilities is $n_1 * n_2 * n_3 * n_4 * n_5 = 10 * 10 * 10 * 10 * 10 = 100,000$. Because all of the zip codes are equally likely, the probability of getting the correct zip code on the first attempt is
\begin{align*}
\frac{1}{100,000}
\end{align*}



\end{document}
%%% Local Variables:
%%% mode: latex
%%% TeX-master: t
%%% End:
